% Define document class
\documentclass[fleqn,usenatbib]{mnras}

\usepackage{newtxtext,newtxmath}
\usepackage[T1]{fontenc}

% Filler text
\usepackage{blindtext}

% These dates will be filled out by the publisher
\date{Accepted XXX. Received YYY; in original form ZZZ}

% Enter the current year, for the copyright statements etc.
\pubyear{2021}

% Title
\title{An open source scientific article}

% Author list
\author{First Author}

% Don't change these lines
\begin{document}
\pagerange{\pageref{firstpage}--\pageref{lastpage}}
\maketitle

% Abstract with filler text
\begin{abstract}
    \blindtext
\end{abstract}

% Main body
\label{firstpage}
\section{Introduction}

Figure~\ref{fig:mandelbrot} was automatically generated using \texttt{showyourwork} with minimal instructions from the user.
By labeling the figure with \verb+\label{fig:mandelbrot}+, the user informs \texttt{showyourwork} that there exists a script in the \texttt{figures} directory called \texttt{mandelbrot.py} that generates the figure files included within that \texttt{figure} environment.
This behavior can be changed or overridden by specifying custom rules in the \texttt{Snakefile}.

% A sample figure
\begin{figure}
    \begin{centering}
        \includegraphics[width=0.4\linewidth]{figures/mandelbrot.pdf}
        \caption{
            A sample plot generated from a \texttt{Python} script in the \texttt{figures} directory, showing the \href{https://en.wikipedia.org/wiki/Mandelbrot_set}{Mandelbrot set}.
            The name of the script is automatically inferred from the figure label in the TeX file.
        }
        % This label tells showyourwork that the script `figures/mandelbrot.py'
        % generates the PDF file included above
        \label{fig:mandelbrot}
    \end{centering}
\end{figure}

% Don't change these lines
\bsp
\label{lastpage}
\end{document}
